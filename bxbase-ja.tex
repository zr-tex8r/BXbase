% 文字コードは UTF-8
% uplatex で組版する
\documentclass[a4paper,uplatex]{jsarticle}
\usepackage{shortvrb}
\MakeShortVerb{\|}
\newcommand{\PkgVersion}{1.1}
\newcommand{\PkgDate}{2017/05/29}
\newcommand{\Pkg}[1]{\textsf{#1}}
\newcommand{\Meta}[1]{$\langle$\mbox{}#1\mbox{}$\rangle$}
\newcommand{\Note}{\par\noindent ※}
\newcommand{\Means}{~:\quad}
\providecommand{\pTeX}{p\TeX}
\providecommand{\upTeX}{u\pTeX}
\providecommand{\pLaTeX}{p\LaTeX}
\providecommand{\upLaTeX}{u\pLaTeX}
%-----------------------------------------------------------
\begin{document}
\title{\Pkg{bxbase} パッケージ\\
(ユーザ命令の解説)}
\author{八登崇之\ (Takayuki YATO; aka.~``ZR'')}
\date{v\PkgVersion \quad[\PkgDate]}
\maketitle

%===========================================================
\section{パッケージの読込}

|\usepackage| で読み込む。オプションは無い。
\begin{quote}\small\begin{verbatim}
\usepackage{bxbase}
\end{verbatim}\end{quote}

\Note 本パッケージのライブラリとしての機能は特定の
エンジンやDVIウェアに依存しないが、以下に述べる命令に
ついては必ずしもそうでないことに注意。

%===========================================================
\section{機能}

%-------------------
\subsection{符号値による文字入力}

\begin{itemize}
\item |\Ux{|\Meta{コード値}|,...}|\\
      |\UI{|\Meta{<コード値>}|,...}|\Means
Unicode コード値による入力を行う。
|\Ux| は欧文用、|\UI| は和文用
(I は Ideographic の意味)。
コード値は以下の形式で表す。
コンマで区切って複数文字入力できる。
\begin{itemize}
\item \Meta{16進数}\Means    |A72C|, |02000B|, |1bd| 等。
\item |+|\Meta{10進数}\Means |+254|, |+0937| 等。
\item |'|\Meta{8進数}\Means  |'376|, |'1651| 等。
\end{itemize}

Unicode 文字の出力には次の順番で利用可能な最初の機能を用いる。
(|\UI| の場合 3) を飛ばす。)
\begin{enumerate}
\item[1)] \Pkg{zxjatype}パッケージ。
この場合、その機構に従って出力される。
すなわち |\UI| は必ず和文フォントで出力し、
|\Ux| は和文/欧文切替の対象となる。
\item[2)] XeTeXのUnicode出力。
\item[3)] (|\Ux| のみ)\Pkg{bxucs}パッケージ。
\item[4)] {\upTeX}の和文Unicode出力。
\item[5)] \Pkg{UTF}/\Pkg{OTF}パッケージ。
\item[6)] \Pkg{ums}/\Pkg{bxsuika}パッケージ。
\end{enumerate}
どれも使えない場合は「16進表現による代替表現」になる。

\item |\AJ{|\Meta{コード値}|,...}|\Means
Adobe-Japan1のコード値による入力を行う。
コード値は10進数で指定する。
出力には次の順番で利用可能な最初の機能を用いる。
\begin{enumerate}
\item[1)] \Pkg{zxotf}パッケージ。
\item[2)] \Pkg{UTF}/\Pkg{OTF}パッケージ。
\end{enumerate}
どれも使えない場合は代替表現になる。

\item |\JI{|\Meta{コード値}|,...}|\Means
いわゆる「JISコード」(JIS~X~0208のGL表現)
のコード値による入力を行う。
コード値指定の方法は |\Ux| と同じ。
出力には{(u)\pTeX}のJISコード和文出力の機能を使う。
使えない場合は代替表現になる。

\item |\KI{|\Meta{コード値}|,...}|\Means
いわゆる「区点コード」のコード値による入力を行う。
コード値指定は「RRCC」(RRは区番号、CCは点番号を10進2桁で表したもの)
の形式で指定する。
出力には{(u)\pTeX}のJISコード和文出力の機能を使う。
使えない場合は代替表現になる。

\item |\bxUx| / |\bxUI| / |\bxAJ| / |\bxJI| / |\bxKI|\Means
|\Ux| 等は非常に短い名前なので他のパッケージと衝突する恐れがあり、
そこでこのパッケージでは既に同名の命令がある場合は上書きしない
ようにしている。
|\bxUx| 等はそれぞれ |\Ux| 等と同じで、
先のような場合にも常に使える。

\end{itemize}

なお、このパッケージは\Pkg{bxutf8}の為のドライバ(|\bxUHex|)
および\Pkg{bxutf8x}の為のドライバを |\Ux| に相当するものに設定する。

%===========================================================
\end{document}
